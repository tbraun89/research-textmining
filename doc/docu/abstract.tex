\begin{abstract}

\noindent Zur Analyse der immer größer werden Datenmengen im Datenmengen wird das Werkzeug des Text-Minings immer wichtiger. Im vorliegenden Forschungsseminar wurde zu diesem Zweck die Verwendbarkeit von Hearst-Patterns zum Aufbau von Hypernym-/Hyponymbeziehungen utnersucht. Im ersten Schritt wurden dazu mit GATE/ANNIE und NLTK zwei vorhandene Bibliotheken verwendet und sich auf Grund der besseren Performanz für NLTK entschieden. Im folgenden wurden erste Hearst-Patterns in NLTK implementiert und erste Testläufe mit diesen durchgeführt. Als Ergebnis dieser ersten Tests lässt sich als Ergebnis festhalten, dass Hearst-Patterns als alleiniges Mittel ungeeignet sind. In einem anschließenden Forschungssemester sollen Mittel erforscht werden diese Ergebnisse zu verfeinern.

\end{abstract}
