\section{Einleitung}

In unserer Arbeit haben wir, im Bereich der linguistischen
Datenverarbeitung, versucht, algorithmisch die semantische
Beziehung der Hyponymie in englisch sprachigen Texten zu erkennen.
Hyponymie beschreibt die semantische Beziehung zwischen zwei Wörtern,
bei denen das Hyponym einem Hyperonym untergeordnet ist, da es eine
spezielle Bedeutung hat. So ist z.B. Apfel ein Hyponym von Obst,
welches dann das Hyperonym ist. \cite{bib:Glueck2005}%
\\%
Um eine solche Untersuchung durchführen zu können, haben wir einen
Textkorpus -- eine große Sammlung von Texten und schriflichen
Aufzeichnungen -- der englischen Sprache benötigt. Dafür haben wir den
\textit{Open American National Corpus} (OANC) verwendet, dieser
enthält in der aktuellen Auflage 11.934.180 Wörter in Form von
wissenschaftlichen Texten, Briefen und schriflichen
Sprachaufzeichnungen. Weitere Informatinen zu OANC bzw. \cite{bib:Ide2002}%
\\%
Zur Umsetzung haben wir das \textit{Natural Language Toolkit}
(NLTK)\footnote{NLTK: \url{http://www.nltk.org/}},
verwendet. Eine Bibliothek für die Programmiersprache Python, welche
eine Sammlung von Akgorithmen und Datenstrukturen bereitstellt, die zur
linguistischen Datenverarbeitung benötigt werden bzw. verwendet werden
können. \cite{bib:Bird2006} Welche Teile der Bibliothek zum Einsatz
kamen, wird später noch genauer erläutert.%