\section{Resultate}%
%
Durch die Auswertung mit unserem DFA haben wir aus den 11.939.180 Wörten
die im OANC enthalten sind, 2.716 Hyperonyme erkennen können, welche
mit 21.028 Hyponymen in Beziehung stehen. Die Erkennung strikt nach
den definierten Hearst Patterns erfolgte, sind auch viele fehlerhafte
Beziehungen enthalten. Da die Analyse nur von Hand durchgeführt werden
kann, haben wir eine Stichprobe\footnote{Die komplette Tabelle
  befindet sich im Anhang.} von 100 Hyperonymen mit 450 Hypoynmen
untersucht und dabei herausgefunden, dass eine Hyponymie nur in ca.
56.22\% tatsächlich stimmt, der Rest sind fehlerhaft erkannte
Beziehungen, somit ist diese Methode alleine nicht brauchbar um
Hyponymie zwischen Wörtern automatisch zu erkennen. Folgend sind einige
Auszüge aus den untersuchten Ergebnissen aufgeführt.%
~\\~\\%
\begin{tabularx}{\textwidth}{|l|X|}
  \hline
  \textbf{Hyperonym} & \textbf{Hyponym} \\
  \hline
  proteins & APC, AlkB, Axin, Bak, Bem1, CD81, LAP1,
  LAP2, MAN1 \\
  \hline
  places & Africa, Amazon, Ankara, California, Indien, \textbf{oh},
  \textbf{New}, \textbf{water} \\
  \hline
  people & Charles, Elvis, Ivan Julius,
  \textbf{pizza}, \textbf{look}, \textbf{that}, \textbf{thinks} \\
  \hline
\end{tabularx}%
~\\~\\~\\%
Dabei sind die fett markierten Wörter fehlerhaft erkannte Hyponyme,
die nicht in Beziehung zu ihrem angegebenen Hyperonym stehen.%
\\%
Um nun bessere Ergebnisse zu erhalten, könnten zum einen die Regeln
der Hearst Patterns angepasst werden, um so schon die Auswahl der
Funde weiter zu begrenzen. Zusätlich könnte noch ein weiterer
Schritt einegführt werden, der die Wahrscheinlichkeit berechnet, dass
ein Hyponym wirklich in Beziehung zu seinem Hyperonym steht. Dafür
müsste entweder ein neuer Algorithmus entwickelt werden, oder ein
passendes Lernverfahren gefunden werden, um die Wahrscheinlichkeit zu
maximieren, dass die gefunden Paare tatsächlich eine Hyponymie
Bezeiehung haben.