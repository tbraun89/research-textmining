\section{Resultate}%
Im Laufe einer ersten Untersuchung unter ausschließlicher Anwendung 
vordefinierter Hearst-Patterns ergab die Auswertung eine korrekte
Erkennung von 56.22\% der untersuchten Stichprobe von 100 Hyperonymen mit
450 zugeordneten Hyponymen. Im Rahmen der Einführung des Lernverfahrens,
wurden die Suche von ausgewählten Hearst-Patterns durch die erlernten 
Schematas ersetzt und diese mit einem initialen Seed auf den Corpus 
angewendet. Aus dem ersten Durchlauf mit gelernten Regeln wurde erneut
eine Stichprobe von 100 Hyperonymen entnommen, denen 208 Hyponyme 
zugeordnet wurden. Hierbei konnte positive Trefferqoute von 58.65\% 
gemessen werden, was minimal besser ist als das Ergebnis der definieren
Hearts-Patterns. Dies ist allerdings nur der Fall, da in das Lernverfahren
eine Blacklist mit nicht zu verwendenden Worten eingepflegt wurde. Ohne
dieses Verfahren waren die Ergebnisse deutlich schlechter, besonders bei
Analysen von Sprachprotollen, wurden etliche Füllwörter bzw. Ausdrücke 
wie ``oh'' und ähnliches erkannt. Nach dem zweiten Iterationsschritt des
Lernverfahrens, war allerdings ein massiver Einbruch der Trefferqoute auf
lediglich 13.02\% korrekter Erkennung zu messen. Hierbei wurde erneut eine
Stichprobe von 100 Hyperonyme analysiert. In diesem Schritt wurden diesen
Hyperonymen, 507 Hyponyme zugeordnet. Der deutliche Abfall der Trefferqoute
ist im besonderen darauf zurückzuführen, das die 41.35\% falschen Ergebnisse 
aus dem ersten Iterationsschritt mit übernommen wurden. In einem 
automatisierten Lernverfahren ist dies aber ohne die Entwicklung eines 
entsprechenden Verifikationsalgorhytmus allerdings nicht zu vermeiden.
Um das Lernverfahren zu verbessern wäre es nun natürlich möglich unter 
Nutzung eines kleinen Corpus nach jedem Iterationsschritt, die gefundenen
Regeln zu überprüfen um die Blacklist zu verbessern und Wörter und Regeln
aus dem Lernverfahren auszuschließen. Eine weitere Möglichkeit wäre die 
Einbeziehung von Wörtern auch vor und hinter dem Bereich zwischen den
untersuchten Hypero- und Hyponymen, da diese durchaus relevant für die
Satzstruktur und den grammatikalischen Zusammenhang seien können.
