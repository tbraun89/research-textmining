\section{Resultate}%
Für die statistische Auswertung der Resultate, wurden jeweils 100
zufällige Hyperonyme, mit ihren zugehörigen Hyponymen, automatisiert
als Stichprobe ausgewählt.
Im Rahmen der Forschung wurde als erster Schritt das erkennen über
vordefinierte Hearst Patterns durchgeführt. Hierbei wurde eine 
korrekte Erkennungsrate von 56.22\% gemessen. Die Stichprobe der 
Hyperonyme enthielt 450 zugehörige Hyponyme.
Diese Ergebnisse wurden zur Verwendung als Trainingsdaten für das 
abgewandelte Pendelverfahren, welches für die weiteren Untersuchungen
verwendet wurde, mit WordNet abgeglichen. Das Resultat dieses Abgleiches
besteht aus 5 Hyperonymen mit 6 dazugehörigen Hyponymen, deren Korrektheit
sicher ist. 

-- LISTE --

Mit diesen Startwerten erlernte das Verfahren daraufhin 2 Regeln für
Hyponomierelationen. 

-- CODE BLOCK --

Durch Anwendung dieser Regeln wurden 744 Hyperonyme mit 1505 
Hyponymen erkannt. Die Untersuchung einer Stichprobe ergab für die 
208 untersuchten Hyponyme eine positive Erkennungsrate von 58.65\%.
Dieses Ergebnis ist minimal besser als dass welches die definierten 
Hearst-Patterns gefunden haben. Hierbei ist allerdings zu beachten,
dass die erlernten Regeln, eine exakte Abbildung eines Hearst-Patterns
und dessen Inverse darstellt. %checK!% 
Im darauf folgenden Iterationsschritt wurden unter Anwendung der gefunden
Hyponomierelationen 156 Regeln erlernt. Mit diesen Regeln wurden 
28.990 Hyperonyme mit 126.386 Hyponymen gefunden. Als Erkennungsrate
der erneuten Stichprobe wurde 13.02\% gemessen. Dieser deutliche 
Abfall in den Ergebnissen lässt darauf schließen, dass der alleinige
Einsatz von selbsterlernten Regeln, welche ähnlich zu Hearst-Patterns
aufgebaut sind, keine ausreichend guten Ergebnisse liefern wird. 
Ohne den Einsatz der Blacklist wären die Ergebnisse sogar noch deutlich
schlechter ausgefallen wie vorherige Tests zeigten. 
Ein weiteres Problem des Verfahrens besteht außerdem in dem exponentiellen
Wachstum der Regeln. In Iterationsschritt 3 wurden nämlich bereits 3.084
Regeln erlernt, was den Erkennungsdurchlauf so rechenaufwendig werden
ließ, das er mit einer Sechskern-CPU die auf 4Ghz getaktet ist, über 
einen Tag gedauert hätte. Um dieses Wachstum zu begrenzen und die 
Erkennungsrate zu verbessern wären nun eine Reihe von Verbesserungen
möglich. Am einfachsten zu realisieren, wäre hierbei wohl eine Ausweitung
der Blacklist durch genaueres Analysieren der zweiten Iteration und
einer Erhöhung der zu findenden Ergebnisse, um eine Regel als korrekt
erlernt zu klassifizieren. So könnten z.B. die bisher festen unteren 
Schranken dynamisch an die gefundenen Regeln angepasst werden und nur
einen bestimmten Prozentsatz der besten Regeln weiter zu verwenden.
Ebenfalls zur Verbesserung der Ergebnisse bzw. zu einer Erweiterung
der Lernmöglichkeiten würde beitragen, nicht nur die Wörter zwischen
Hyperonym und Hypernym zu betrachten, sondern auch Wörter, welche
vor und nach den entsprechenden Signalwörtern in die Regeln,
einzubeziehen, da diese durchaus relevant für den Satzinhalt bzw.
die grammatikalischen Zusammenhänge sein kann. 


