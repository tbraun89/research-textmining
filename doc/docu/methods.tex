\section{Methoden}
\subsection{Part of Speach Tagging}
Als Vorarbeit für die Hearst Patterns ist es nötig den Text per Part of Speach Tagging (PoST) zu klassifizieren.
Hierbei wird jedem Wort im Text mit Hilfe von statistischen Wahrscheinlichkeiten oder einem Satz von Regeln eine Klasse zugewiesen.
In der vorliegenden Arbeit wurde hierfür der Brill Tagger genutzt. Dieser besitzt eine Liste von bekannten Worten mit zugehörigen
Wahrscheinlichkeiten ihrer Klassenzugehörigkeit. In einem ersten Schritt wird allen bekannten Worten nun die Klasse mit der höchsten
Standartwahrscheinlichkeit zugewiesen und alle unbekannten Worte werden an Hand von weiteren Regeln klassiefiziert. Großgeschriebene 
unbekannte Wörter werden provisorisch als Nomen klassifiziert und Wörter mit dem Prefix \textit{ous} werden als Adjektive eingestuft,
da die meisten der bekannten Wörter mit diesem Prefix als Adjektive klassifiziert sind. 

\subsection{Analyse mit Hilfe von Hearst Patterns}

Zur Analyse der vorliegenden Daten wurde die Methode der \textit{Hearst Patterns} verwendet. 
Dies ermöglicht es, für ein gegebens \textit{Hypernym} ein oder mehrere \textit{Hyponyme} zu finden. 
Dabei bestehen für die Hyponymie folgende Regeln wenn M die Menge aller Wörter der 
entsprechenden Sprache ist und R eine Relation zwischen zwei Wörtern dieser Sprache, wie in Beziehung \ref{eq:relation} dargestellt.
Diese Relation besitzt folgende Eigenschaften wie in der Ausarbeitung von \cite{bib:Snow2004} beschrieben.

\begin{equation}
  \label{eq:relation}
  R \subseteq M \times M
\end{equation}

\begin{equation}
  \label{eq:trans}
  \forall x, y, z \in M : xRy \land yRz \Rightarrow xRz
\end{equation}

\begin{equation}
  \label{eq:irref}
  \forall x \in M : \neg xRx
\end{equation}

\begin{equation}
  \label{eq:assym}
  \forall x, y \in M : xRy \Rightarrow \neg (yRx)
\end{equation}

Beziehung \ref{eq:trans} beschreibt die Transitivität, ein zu einem Hypernym gehörendes Hyponym, 
kann also selbst wieder ein Hypernym für ein oder mehrere andere Hyponyme sein.
Ein Hyponym kann allerdings nicht sein eigenes Hypernym sein wie in Beziehung \ref{eq:irref} gezeigt und ebenfalls darf es
nicht symmetrisch sein, also nicht sowohl eine Hypernym- als auch Hyponymbeziehung zu einem anderen Wort
der Menge M besitzen \ref{eq:assym}.
Um die entsprechenden Hypernyme und Hyponyme im Corpus zu finden, werden vordefinierte Muster, die sogenannten Hearst Patterns
verwendet. Die folgende Liste zeigt einige der von \cite{bib:Snow2004} empfohlenen Hearst Patterns, die auch in der erstellten
Implementierung verwendet werden. Hierbei ist $NP_{X}$ das Hyponym und $NP_{Y}$ das dazugehörige Hypernym.

\begin{itemize}
\item $NP_{X}$ and other $NP_{Y}$
\item $NP_{X}$ or other $NP_{Y}$
\item $NP_{Y}$ such as $NP_{X}$
\item Such $NP_{Y}$ as $NP_{X}$
\item $NP_{Y}$ including $NP_{X}$
\item $NP_{Y}$, especially $NP_{X}$
\end{itemize}

Zur Umsetzung der Hearst Patterns wird in der entwickelten Implementierung\footnote{\href{https://github.com/tbraun89/research-textmining}{Github:
 https://github.com/tbraun89/research-textmining}} ein deterministischer finiter Automat (DFA) eingesetzt. 
