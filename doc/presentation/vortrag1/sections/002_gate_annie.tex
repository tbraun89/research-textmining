\section{NE recognition}
\label{sec:ne-recognition}

\begin{frame}
  \frametitle{NE recognition}

  \begin{itemize}
  \item Text wird in Tokens unterteilt
  \item Vergleich mit Gazetteer Listen
  \item Anwendung eines Sentence Splitters
  \item POS-Tagger 
  \end{itemize}
\end{frame}

\begin{frame}
  \frametitle{Tokenizer}
  
  \begin{itemize}
  \item Token sind grundlegende Textbestandteile wie
    \begin{itemize}
      \item Wörter (groß/klein Schreibung)
      \item Zahlen
      \item Satzzeichen
      \item Sonderzeichen
    \end{itemize}
  \end{itemize}
\end{frame}

\begin{frame}
  \frametitle{Gazetteer/Ortslexikon}

  \begin{itemize}
  \item beinhaltet Informationen zu Orten wie Einwohnerzahlen, Koordinaten etc.
  \item bietet Basisinformationen zu Begin der Analyse
  \item Beispiel Dresden:
    \begin{itemize}
    \item Bundesland: Sachsen
    \item Einwohner: 525.105
    \item Koordinaten: 51Grad 3Min Nord, 13 Grad 44 Min Ost
    \end{itemize}
  \end{itemize}
\end{frame}

\begin{frame}
  \frametitle{Sentence Splitter}
  \begin{itemize}
  \item Unterteilt Sätze in Teilinformationen
  \item nutzt dazu festgelegte Regeln
  \item Eine Regel für Stadionnamen würde z.B. so aussehen: 
    \begin{itemize}
    \item NNP, groß geschrieben, steht links von ``Stadium''
    \item Treffer: Wembley Stadium, Emirates Stadium
    \item kein Treffer: White Hart Lane, Stamford Bridge
    \end{itemize}
  \end{itemize}
\end{frame}

\begin{frame}
  \frametitle{Part of Speech Tagger}
  \begin{itemize}
  \item Unterteilt die gefunden Wörter in spezielle Klassen (Nomen, Verben, Adjektive...)
  \item Es existieren verschiedene Tagger mit unterschiedlicher Effizienz (Brown Tagger, Maximum Entropie Tagger, ...)
  \item Die meisten Tagger nutzen Hidden Markov Modelle
  \item GATE nutzt den Brill Tagger
  \end{itemize}
\end{frame}

\begin{frame}
  \frametitle{Brill Tagger}
  \begin{itemize}
  \item Entwickelt von Eric Brill im Jahr 1995
  \item Algorhytmus benötigt Überwachtes Lernen
  \item Der Algorhytmus:
    \begin{itemize}
    \item Während der Initialisierung werden bekannte Wörter mit dem wahrscheinlichsten Tag versehen 
    \item Unbekannten Wörtern, werden Tags nach Merkmalen wie Groß-/Kleinschreibung oder Präfixen vergeben
    \item Danach werden kleine Textteile solange im Zusammenhang auf eine Reihe von Regeln überprüft bis keine Regeln mehr vorhandensind oder ein Schwellenwert erreicht ist
    \item Dabei gilt immer der letzte gefunde Tag als der korrekte Tag für das Wort, die vorherigen Tags werden ersetzt
    \end{itemize}
  \end{itemize}
\end{frame}

