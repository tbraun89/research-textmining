\section{NE recognition}
\label{sec:ne-recognition}

\begin{frame}
  \frametitle{NE recognition}

  \begin{itemize}
  \item Text wird in Tokens unterteilt
  \item Vergleich mit Gazetteer Listen
  \item Anwendung eines Sentence Splitters
  \item POS-Tagger 
  \end{itemize}
\end{frame}

\begin{frame}
  \frametitle{Tokenizer}
  
  \begin{itemize}
  \item Token sind grundlegende Textbestandteile wie
    \begin{itemize}
      \item Wörter (Verben, Nomen, Adjektive...)
      \item Zahlen
      \item Satzzeichen
    \end{itemize}
  \end{itemize}
\end{frame}

\begin{frame}
  \frametitle{Gazetteer/Ortslexikon}

  \begin{itemize}
  \item beinhaltet Informationen zu Orten wie Einwohnerzahlen, Koordinaten etc.
  \item bietet Basisinformationen zu Begin der Analyse
  \item Beispiel Dresden:
    \begin{itemize}
    \item Bundesland: Sachsen
    \item Einwohner: 525.105
    \item Koordinaten: 51Grad 3Min Nord, 13 Grad 44 Min Ost
    \end{itemize}
  \end{itemize}
\end{frame}

\begin{frame}
  \frametitle{Sentence Splitter}
  \begin{itemize}
  \item Unterteilt Sätze in Teilinformationen
  \item nutzt dazu festgelegte Regeln
  \item Eine Regel für Stadionnamen würde z.B. so aussehen: 
    \begin{itemize}
    \item NNP, groß geschrieben, steht links von ``Stadium''
    \item Treffer: Wembley Stadium, Emirates Stadium
    \item kein Treffer: White Hart Lane, Stamford Bridge
    \end{itemize}
  \end{itemize}
\end{frame}

\begin{frame}
  \frametitle{Part of Speech Tagger}
  \begin{itemize}

  \end{itemize}
\end{frame}
