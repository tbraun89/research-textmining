\section{Gate und ANNIE}

Gate (General Architecture for Text Engineering) ist ein Tool welches
von der \textit{University of Sheffield} als freie Software entwickelt
und eine erweiterebare Komponente zum analysieren und verarbeiten von
Texten in natürlicher Sprache bietet.
\cite{Cunningham:2000}

ANNIE (A Nearly-New Information Extraction System) ist ein Plugin für
Gate mit dem es möglich ist einen Corpus zu klassifizieren. Dafür
besitzt es einen Tokeniser, Sentence splitter, Part-of-speech tagger,
gezetterer und finite sentence transducter.
\cite{Cunningham:2002}

Für unsere Implementierung haben wir eine neue Klasse \textbf{ANNIE}
angelegt, die das Plugin initialisiert. Gate muss dafür bereits in der
Hauptroutine geladen werden, dabei muss der Pfad zu den Gate Resourcen
und den Plugins angegeben werden. ANNIE wurde so implementiert, dass
beim Erzeugen einer neuen Instanz der Corpus als \lstinline{String}
übergeben werden kann. Dieser wird dann analysiert und kann in dann
über eine Methode als annotierter Text zurückgegeben werden. 
% TODO Infos wie der Text annotiert wird hinzufügen 

Die Generierung des \lstinline{StructuredDocument} mit Annotationen im
Corpus, nachdem die Verarbeitung von ANNIE abgeschlossen wurde hat
eine Laufzeit von

$$\mathcal{O}(2a \cdot log(a))$$ % TODO Beweis? Korrektur?

wobei $a$ die Anzahl der für die Rückgabe relevanten Annotationen ist.


\section{Hearst Patterns}

Herst Patterns werden verwendet um \textit{Hyponyme} für ein gegebenes
\textit{Hyperonym} zu finden. Dabei gelten folgende Regeln

$$R \subseteq M \times M$$
$$\forall x, y, z \in M : xRy \land yRz \Rightarrow xRz$$
$$\forall x \in M : \neg xRx$$
$$\forall x, y \in M : xRy \Rightarrow \neg yRx$$

wobei $M$ die Menge aller Wörter der entsprechenden Sprache ist.
% TODO Quelle hinzufügen (aktuell Wikipedia ~> dort keine Quelle
% angegben -.-)

Um die entsprechenden Hyponyme in einem Corpus zu finden, werden
bestimmte Muster in Texten gesucht, die folgende Liste zeigt die
verwendeten Patterns, dabei ist $NP_{X}$ das Hyponym und $NP_{Y}$ das
dazugehörige Hyperonym.

\begin{itemize}
\item $NP_{X}$ and other $NP_{Y}$
\item $NP_{X}$ or other $NP_{Y}$
\item $NP_{Y}$ such as $NP_{X}$
\item Such $NP_{Y}$ as $NP_{X}$
\item $NP_{Y}$ including $NP_{X}$
\item $NP_{Y}$, especially $NP_{X}$
\end{itemize}

So können Hyponyme zu entsprechenden Hyperonymen gefunden werden.
\cite{Snow:2004}

In unserer Implementierung verwenden wir reguläre Ausdrücke um nach
den Mustern zu suchen und gleichen dies zunächst mit den Ergebnissen
von ANNIE ab. So kann zuerst sichergestellt werden, dass es sich um
zwei gefundene Nomen handelt. Zudem müssen alle beschriebenen Regeln
eingehalten werden. Ist dies der Fall kann die Anwendung ein neues
Hyponym und falls noch nicht vorhanden ein neues Hyperonym lernen.


\section{WordNet}

% TODO Wordnet zum validieren der Hyperonyme und Hyponyme ~> siehe Snow:2004